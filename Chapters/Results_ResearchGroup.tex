% Chapter discussion
\chapter{Results of the Research Group}%-5/10pagine} % Main chapter title

%\label{Chapter3} % For referencing the chapter elsewhere, use \ref{Chapter4} 

%OK ENZA OK ERIK ----------------------------------------------------------------------------------------
The project presented here is part of a new line of research under development in the host research group at the Institute of Genetics and Biophysics of the National Research Council (IGB-CNR).
Due to this work, the team at IGB-CNR where I completed my master thesis project has recently started a collaboration \cite{eizenga2020succinct} with a team at the University of California Santa Cruz that has outstanding expertise in graphical pangenomic models \cite{garrison2019graphical,hickey2020genotyping,eizenga2020pangenome,siren2020haplotype,rhie2020towards,eizenga2020succinct,martiniano2020removing,llamas2019strategy,carletti2019graph,siren2019supplement,chin2019diploid,garrison2018variation,garg2018graph,paten2018superbubbles,computational2018computational}.
Although the group at UCSC has great expertise in methods for pangenome construction, manipulation, and the generalization of basic bioinformatic processes to these models, they have not extensively focused on population genetic questions.
This suggests an opportunity for future research, and we hope to continue this line of collaboration with this and other groups interested in these topics.

%Descrive il progetto di ricerca di cui fa parte l’attività sperimentale svolta dal candidato. Vanno qui riportati i risultati ottenuti dal gruppo di ricerca in precedenza e/o in contemporanea rispetto all’attività svolta dal candidato che siano rilevanti per l’argomento della tesi. Nel caso in cui il candidato abbia svolto, nell’ambito del gruppo di appartenenza, un progetto di ricerca nuovo e autonomo, questa sezione può essere eliminata. In questa sezione è richiesto l’uso di illustrazioni.