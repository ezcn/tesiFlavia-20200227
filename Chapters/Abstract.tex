% Chapter discussion

\chapter{Abstract} %-1pagina}
    
 % Main chapter title

\label{Chapter0} % For referencing the chapter elsewhere, use \ref{Chapter4} 

%----------------------------------------------------------------------------------------

% Define some commands to keep the formatting separated from the content 
\newcommand{\keyword}[1]{\textbf{#1}}
\newcommand{\tabhead}[1]{\textbf{#1}}
\newcommand{\code}[1]{\texttt{#1}}
\newcommand{\file}[1]{\texttt{\bfseries#1}}
\newcommand{\option}[1]{\texttt{\itshape#1}}

%~~~~~~~~~~~~~~~~~~~~~~~~~~~~~~~~~~~~~~~~~~~~~~~~~~~~~~~~~~~~~~~~~~~~~~~~~~~~~~~~~~~~~~~
Studies of genomics typically assume a single linear reference genome and these presents limits. I started the development of a software library for the statistical analysis using pangenomic data models, these represent whole genome alignments in a compact structure, without loosing information. Because a pangenome embeds the linear genomes from which it is constructed, we can choose a particular reference genome and project the variants on it(which appear as bubbles in the graph). Therefore, I focused on develop algorithms for bubble detection, generating Variant Call Format (VCF) files directly from graphs.

\noindent
We can use this variants projection to drive standard population genetic analyses. In my thesis, I focused on the calculation of Fst for three different time.

