% Chapter discussion

\chapter{Abstract} %-1pagina}
    
 % Main chapter title

\label{Chapter0} % For referencing the chapter elsewhere, use \ref{Chapter4} 

%----------------------------------------------------------------------------------------

% Define some commands to keep the formatting separated from the content 
\newcommand{\keyword}[1]{\textbf{#1}}
\newcommand{\tabhead}[1]{\textbf{#1}}
\newcommand{\code}[1]{\texttt{#1}}
\newcommand{\file}[1]{\texttt{\bfseries#1}}
\newcommand{\option}[1]{\texttt{\itshape#1}}

%~~~~~~~~~~~~~~~~~~~~~~~~~~~~~~~~~~~~~~~~~~~~~~~~~~~~~~~~~~~~~~~~~~~~~~~~~~~~~~~~~~~~~~~
Studies of genomes typically assume a single linear reference genome. This can make it difficult to observe sequences in other genomes that are divergent from the reference, which in turn can limit the accuracy and completeness of our analyses. To address this limitation, we can work with pangenome reference systems that represent the mutual relationships between many complete genomes without using a single one as a point of reference. These models often encode genomes and their mutual alignment in a compact graphical structure. Here, I begin the development of a software library for the statistical analysis of population genetic parameters of genomes stored in these graphical pangenomic data models. Because a pangenome embeds the linear genomes from which it is constructed, we can choose a particular reference genome and project the variants on it(which appear as bubbles in the graph). Therefore, I focused on develop algorithms for bubble detection, generating Variant Call Format (VCF) files directly from graphs. I then developed methods to annotate these variants with allele frequencies of genomes embedded in the input graph, and then explored the calculation of Fst.

\vspace{1.00cm}


Gli studi sui genomi in genere assumono un singolo genoma di riferimento lineare. Ciò può rendere difficile l'osservazione di sequenze in altri genomi divergenti dal riferimento, che a loro volta possono limitare l'accuratezza e la completezza delle nostre analisi. Per ovviare a questa limitazione, possiamo lavorare con modelli pangenomici che rappresentano le relazioni reciproche tra molti genomi completi senza utilizzarne uno come punto di riferimento. Questi modelli spesso codificano i genomi e il loro reciproco allineamento in una struttura grafica compatta. In questo caso, inizio lo sviluppo di una libreria software per l'analisi statistica dei parametri genetici di popolazione dei genomi memorizzati in questi modelli di dati pangenomici. Poiché un pangenoma incorpora i genomi lineari da cui è costruito, possiamo scegliere un particolare genoma di riferimento e proiettare le varianti su di esso (che appaiono come bolle nel grafico). Pertanto, mi sono concentrata sullo sviluppo di algoritmi per il rilevamento di bolle, la generazione di file VCF (Variant Call Format) direttamente dai grafi. Ho quindi sviluppato metodi per annotare queste varianti con frequenze alleliche di genomi incorporate nel grafico di input, e quindi ho esplorato il calcolo di Fst.





%\noindent
%We can use this variants projection to drive standard population genetic analyses. In my thesis, I focused on the calculation of Fst for three different time.

